\documentclass{article}
\usepackage{pythontex}
\usepackage{glossagen}

\begin{document}


\section{Abstract}
Abstract
Self-driving laboratories (SDLs) promise an accelerated application of the scientific method. Through the automation of high-throughput experimentation, and autonomization of experiment planning and execution, SDLs hold the potential to greatly accelerate research in chemistry and materials discovery. This review article provides an in-depth analysis of the state-of-the-art in SDL technology, its applications across various scientific disciplines, and the potential implications for research, and industry. This review additionally provides an overview of the enabling technologies for SDLs, including their hardware, software, and integration with laboratory infrastructure. Most importantly, this review explores the diverse range of scientific domains where SDLs have made significant contributions, from drug discovery and materials science to genomics and chemistry. We provide a comprehensive review of existing real-world examples of SDLs, their different levels of automation, and the challenges and limitations associated with each domain.
1. Introduction
In the face of pressing global challenges such as climate change, energy sustainability, and current or emerging healthcare crises, we must seek efficient solutions in the context of a growing global population and increasing resource demands. The accelerated development of materials, technology, and scientific understanding emerges as a potential avenue for tackling these challenges. Traditional research methods, characterized by gradual progress with limited efficiency, may prove insufficient for the urgency these challenges demand. Self-driving laboratories (SDLs), integrating laboratory automation and data-driven decision making, can potentially facilitate a more rapid and efficient exploration of solutions, while offering multiple advantages over traditional scientific discovery. Notably, automated experimentations can perform experiments faster and with higher precision, while data-driven search algorithms can quickly and efficiently explore experimental space based on feedback from available data (“closed-loop” experimentation). Additionally, issues such as reproducibility challenges and the underrepresentation of negative results in the scientific literature have been identified.1–4 The advent of SDLs encourages the further digitization of research.5 The utilization of automated systems enables more precise documentation of experimental protocols, enhancing repeatability and reproducibility, while digitization facilitates data recording and sharing, with particular emphasis on the significance of negative or null results, contributing to a more comprehensive and accurate portrayal of scientific endeavors. High quality large datasets made possible by autonomous experimentation would aid in the development of artificial intelligence (AI) for materials science and chemistry, creating better machine learning (ML) and deep learning (DL) models, and enhancing the decision-making capabilities of SDLs.
The primary objectives of SDLs in the context of chemistry and materials science encompass the optimization of materials or processes and the facilitation of novel discoveries. The concept of SDLs has two critical defining dimensions: software autonomy and hardware autonomy. Hence, we classify SDLS according to the degrees of autonomy in these two axes.
Concerning software autonomy, which pertains to experiment selection, SDLs can be classified into three categories: (1) single iteration of automated experimentation with a data-driven method to select the next possible experiments, (2) multiple iterations on a closed-loop system, in which the experimental results feedback to guide another round of automated experiments, and (3) generative approaches, in which multiple iterations of closed-loop optimization are performed in a search space or chemical space that is generated by an algorithm.


\section{Introduction}
Your introduction text here...

% \glossary % non functional yet :-(

\begin{pycode}
from glossagen.pipelines.latex_glossary import main
main('./latex.tex')
\end{pycode}

\section*{Bibliography}
blablabalba

\end{document}
